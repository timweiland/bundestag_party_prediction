\documentclass{article}

% if you need to pass options to natbib, use, e.g.:
%     \PassOptionsToPackage{numbers, compress}{natbib}
% before loading neurips_2021

% ready for submission
\usepackage[preprint]{neurips_2021}

% to compile a preprint version, e.g., for submission to arXiv, add add the
% [preprint] option:
%     \usepackage[preprint]{neurips_2021}

% to compile a camera-ready version, add the [final] option, e.g.:
%     \usepackage[final]{neurips_2021}

% to avoid loading the natbib package, add option nonatbib:
%    \usepackage[nonatbib]{neurips_2021}

\usepackage[utf8]{inputenc} % allow utf-8 input
\usepackage[T1]{fontenc}    % use 8-bit T1 fonts
\usepackage[colorlinks=true]{hyperref}       % hyperlinks
\usepackage{url}            % simple URL typesetting
\usepackage{booktabs}       % professional-quality tables
\usepackage{amsfonts}       % blackboard math symbols
\usepackage{nicefrac}       % compact symbols for 1/2, etc.
\usepackage{microtype}      % microtypography
\usepackage{xcolor}         % colors

\title{Predicting Political Parties\\ from Text}

% The \author macro works with any number of authors. There are two commands
% used to separate the names and addresses of multiple authors: \And and \AND.
%
% Using \And between authors leaves it to LaTeX to determine where to break the
% lines. Using \AND forces a line break at that point. So, if LaTeX puts 3 of 4
% authors names on the first line, and the last on the second line, try using
% \AND instead of \And before the third author name.

\author{%
  Mattia Masiero\\
  Matrikelnummer 6100692\\
  \texttt{mattia.masiero@student.uni-tuebingen.de} \\
  \And
  Tim Weiland\\
  Matrikelnummer 6010188\\
  \texttt{tim.weiland@student.uni-tuebingen.de} \\
}

\begin{document}

\maketitle

\begin{abstract}
  We are planning to use \href{https://de.openparliament.tv/}{a collection of Bundestag speeches held since 2017} to see how well the association to a political party can be predicted from text using simple features such as PCA applied to bag-of-words, text/sentence length or width of vocabulary. We plan on using logistic regression for the classification. To analyze if and how speeches have changed over time, we plan on combining the OpenParliament dataset with another dataset such as \href{https://dataverse.harvard.edu/dataset.xhtml?persistentId=doi:10.7910/DVN/E4RSP9}{ParlSpeech} or the \href{https://politische-reden.eu/}{German Political Speeches Corpus}. If time permits, we would like to use the results of our analysis to create a political slogan generator.
\end{abstract}

\end{document}
